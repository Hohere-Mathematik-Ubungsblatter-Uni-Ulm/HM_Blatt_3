\section{Aufgabe 11}
Wir müssen zuerst $2-i2\sqrt{3}$ in die Form $re^{i\phi}$ Umrechnen
$$r = \sqrt{2^2 - i^2 (2\sqrt{3})^2} = \sqrt{4+(2\sqrt{3})^2}=\sqrt{4+4\cdot3}=\sqrt{4+12}=\sqrt{16}=4$$
$$\phi = arctan(\frac{-2\sqrt{3}}{2}) = arctan(-\sqrt{3}) = -\frac{\pi}{3}$$
$$\Rightarrow 2-i2\sqrt{3} = 4exp(\frac{-i\pi}{3})$$
Wir berechnen nun den Betrag und das Argument der füfnten Wurzeln: 
$$r^{\frac{1}{4}}=4^{\frac{1}{5}}$$
und
$$arg=\frac{-\pi}{3 \cdot 5}+\frac{2\pi k}{5} \text{ für } k=0,1,2,3,4$$
$\Rightarrow$ Jede fünfte Wurzel hat die Form $z_k = 4^{\frac{1}{5}}e^{i(\frac{-\pi}{15}+\frac{2\pi k}{5})}$ für $k=0,1,2,3,4$. Also Rotieren wir den Winkel um $\frac{2\pi}{5}$ für jede Nachfolgende Wurzel
$$k=0 \Rightarrow \text{fünfte Wurzel: } 4^{\frac{1}{5}}e^{i(\frac{-\pi}{15})}$$
$$k=1 \Rightarrow \text{fünfte Wurzel: } 4^{\frac{1}{5}}e^{i(\frac{-\pi}{15}+\frac{2\pi}{5})}$$
$$k=2 \Rightarrow \text{fünfte Wurzel: } 4^{\frac{1}{5}}e^{i(\frac{-\pi}{15}+\frac{4\pi}{5})}$$
$$k=3 \Rightarrow \text{fünfte Wurzel: } 4^{\frac{1}{5}}e^{i(\frac{-\pi}{15}+\frac{6\pi}{5})}$$
$$k=4 \Rightarrow \text{fünfte Wurzel: } 4^{\frac{1}{5}}e^{i(\frac{-\pi}{15}+\frac{8\pi}{5})}$$