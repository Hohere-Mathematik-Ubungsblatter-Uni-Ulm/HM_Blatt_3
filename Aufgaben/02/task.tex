\section{Aufgabe 2}
Sei $a \neq 0$
$$\Rightarrow ax^2+bx+c = 0$$
$$\Leftrightarrow ax^2+bx = -c$$
$$\Leftrightarrow x^2 + \frac{b}{a}x = - \frac{c}{a}$$
$$\Leftrightarrow x^2 + \frac{b}{a}x + (\frac{b}{2a})^2 = -\frac{c}{a} + (\frac{b}{2a})^2 $$
$$\Leftrightarrow (x+\frac{b}{2a})^2 = \frac{b^2}{4a^2} - \frac{c\cdot 4a}{a\cdot 4a}$$
$$\Leftrightarrow (x+\frac{b}{2a})^2 = \frac{b^2}{4a^2} - \frac{c\cdot 4a}{4a^2}$$
$$\Leftrightarrow (x+\frac{b}{2a})^2 = \frac{b^2-4ac}{4a^2}$$
$$\Leftrightarrow x+\frac{b}{2a} = \pm \sqrt{\frac{b^2-4ac}{4a^2}}$$
$$\Leftrightarrow x = \frac{-b}{2a} \pm \sqrt{\frac{b^2-4ac}{4a^2}}$$
$$\Leftrightarrow x_{1, 2} = \frac{-b}{2a} \pm \frac{\sqrt{b^2-4ac}}{2a}$$
$$\Leftrightarrow x_{1, 2} = \frac{-b \pm \sqrt{b^2-4ac}}{2a}$$
Da falls $a = 0$ das Polynom nicht mehr ein Polynom des zweiten Grades ist, fällt dieser Fall im Normalfall weg. Allerdings wurde ich darauf hingewiesen dass wir dies trotzdem herleiten sollen. \\
Sei $a = 0$
$$\Rightarrow ax^2+bx+c = 0$$
$$\Leftrightarrow bx+c = 0$$
$$\Leftrightarrow bx = -c$$
$$\Leftrightarrow x = -\frac{b}{c}$$